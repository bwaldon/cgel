\documentclass[12pt]{standalone}
\usepackage{times}
\usepackage{parsetree}
\usepackage{textcomp}
\pagestyle{empty}
%----------------------------------------------------------------------
% Node labels in CGEL trees are defined with \NL, which is defined so that
% \NL{Abcd}{Xyz} yields a label with the function Abcd on the top, in
% sanserif font, followed by a colon, and the category Xyz on the bottom.
\newcommand{\NL}[2]{\begin{tabular}[t]{c}\small\textsf{#1:}\\
#2\end{tabular}}
%----------------------------------------------------------------------
\begin{document}
\begin{parsetree}
	%why is there a word for everything
(.Clause.
	(.\NL{Prenucleus}{AdvP\textsubscript{i}}. 
	(.\NL{Head}{Adv}.  `why' ))
	(.\NL{Head}{Clause}. 
	(.\NL{Prenucleus}{V\textsubscript{j}}. `is')
	(.\NL{Head}{Clause}. 
	(.\NL{Subj}{NP}. 
	(.\NL{Subj}{Nom}. 
	(.\NL{Head}{N}. `there' )))
	(.\NL{Head}{VP}. 
	(.\NL{Head}{GAP\textsubscript{j}}.  `--' )
	(.\NL{PredComp}{NP}. 
	(.\NL{Head}{DP}. 
	(.\NL{Head}{D}.  `a' ))
	(.\NL{Head}{Nom}. 
	(.\NL{Head}{N}.  `word' )
	(.\NL{Mod}{PP}. 
	(.\NL{Head}{P}.  `for' )
	(~.\NL{Obj}{NP}. `everything')
	))))
	(.\NL{Adjunct}{GAP\textsubscript{i}}.  `--' )
	)))
	
\end{parsetree}
\end{document}
